\chapter{\ac{RF}}
\section{Ingénierie des caractéristiques}
\textit{feature engineering}
\subsection{Charge}
\subsection{Décharge}
\section{Procédé}
\section{Fusion}
La principe de \emph{late fusion} a été utilisé pour construire un modèle qui prend en compte les prédictions de charge et de décharge. La raison est d'ordre pratique: la prise en compte des données de décharge étant optionnelle, un modèle de prédiction complet a d'abord été construit pour la charge puis a été adapté pour la décharge.

\begin{itemize}
    \item Prédiction de la classe (qualité) pour chaque charge (31 prédictions).
    \item Prédiction de la classe (qualité) pour chaque charge (28 prédictions).
\end{itemize}

Pour fusionner les prédiction de charge et d'écharge, on s'aide de leur ID (charge-nb et discharge-nb)

\begin{enumerate}
	\item 
	\item 
	\item 
\end{enumerate}

